
% Default to the notebook output style

    


% Inherit from the specified cell style.




    
\documentclass[11pt]{article}

    
    
    \usepackage[T1]{fontenc}
    % Nicer default font (+ math font) than Computer Modern for most use cases
    \usepackage{mathpazo}

    % Basic figure setup, for now with no caption control since it's done
    % automatically by Pandoc (which extracts ![](path) syntax from Markdown).
    \usepackage{graphicx}
    % We will generate all images so they have a width \maxwidth. This means
    % that they will get their normal width if they fit onto the page, but
    % are scaled down if they would overflow the margins.
    \makeatletter
    \def\maxwidth{\ifdim\Gin@nat@width>\linewidth\linewidth
    \else\Gin@nat@width\fi}
    \makeatother
    \let\Oldincludegraphics\includegraphics
    % Set max figure width to be 80% of text width, for now hardcoded.
    \renewcommand{\includegraphics}[1]{\Oldincludegraphics[width=.8\maxwidth]{#1}}
    % Ensure that by default, figures have no caption (until we provide a
    % proper Figure object with a Caption API and a way to capture that
    % in the conversion process - todo).
    \usepackage{caption}
    \DeclareCaptionLabelFormat{nolabel}{}
    \captionsetup{labelformat=nolabel}

    \usepackage{adjustbox} % Used to constrain images to a maximum size 
    \usepackage{xcolor} % Allow colors to be defined
    \usepackage{enumerate} % Needed for markdown enumerations to work
    \usepackage{geometry} % Used to adjust the document margins
    \usepackage{amsmath} % Equations
    \usepackage{amssymb} % Equations
    \usepackage{textcomp} % defines textquotesingle
    % Hack from http://tex.stackexchange.com/a/47451/13684:
    \AtBeginDocument{%
        \def\PYZsq{\textquotesingle}% Upright quotes in Pygmentized code
    }
    \usepackage{upquote} % Upright quotes for verbatim code
    \usepackage{eurosym} % defines \euro
    \usepackage[mathletters]{ucs} % Extended unicode (utf-8) support
    \usepackage[utf8x]{inputenc} % Allow utf-8 characters in the tex document
    \usepackage{fancyvrb} % verbatim replacement that allows latex
    \usepackage{grffile} % extends the file name processing of package graphics 
                         % to support a larger range 
    % The hyperref package gives us a pdf with properly built
    % internal navigation ('pdf bookmarks' for the table of contents,
    % internal cross-reference links, web links for URLs, etc.)
    \usepackage{hyperref}
    \usepackage{longtable} % longtable support required by pandoc >1.10
    \usepackage{booktabs}  % table support for pandoc > 1.12.2
    \usepackage[inline]{enumitem} % IRkernel/repr support (it uses the enumerate* environment)
    \usepackage[normalem]{ulem} % ulem is needed to support strikethroughs (\sout)
                                % normalem makes italics be italics, not underlines
    

    
    
    % Colors for the hyperref package
    \definecolor{urlcolor}{rgb}{0,.145,.698}
    \definecolor{linkcolor}{rgb}{.71,0.21,0.01}
    \definecolor{citecolor}{rgb}{.12,.54,.11}

    % ANSI colors
    \definecolor{ansi-black}{HTML}{3E424D}
    \definecolor{ansi-black-intense}{HTML}{282C36}
    \definecolor{ansi-red}{HTML}{E75C58}
    \definecolor{ansi-red-intense}{HTML}{B22B31}
    \definecolor{ansi-green}{HTML}{00A250}
    \definecolor{ansi-green-intense}{HTML}{007427}
    \definecolor{ansi-yellow}{HTML}{DDB62B}
    \definecolor{ansi-yellow-intense}{HTML}{B27D12}
    \definecolor{ansi-blue}{HTML}{208FFB}
    \definecolor{ansi-blue-intense}{HTML}{0065CA}
    \definecolor{ansi-magenta}{HTML}{D160C4}
    \definecolor{ansi-magenta-intense}{HTML}{A03196}
    \definecolor{ansi-cyan}{HTML}{60C6C8}
    \definecolor{ansi-cyan-intense}{HTML}{258F8F}
    \definecolor{ansi-white}{HTML}{C5C1B4}
    \definecolor{ansi-white-intense}{HTML}{A1A6B2}

    % commands and environments needed by pandoc snippets
    % extracted from the output of `pandoc -s`
    \providecommand{\tightlist}{%
      \setlength{\itemsep}{0pt}\setlength{\parskip}{0pt}}
    \DefineVerbatimEnvironment{Highlighting}{Verbatim}{commandchars=\\\{\}}
    % Add ',fontsize=\small' for more characters per line
    \newenvironment{Shaded}{}{}
    \newcommand{\KeywordTok}[1]{\textcolor[rgb]{0.00,0.44,0.13}{\textbf{{#1}}}}
    \newcommand{\DataTypeTok}[1]{\textcolor[rgb]{0.56,0.13,0.00}{{#1}}}
    \newcommand{\DecValTok}[1]{\textcolor[rgb]{0.25,0.63,0.44}{{#1}}}
    \newcommand{\BaseNTok}[1]{\textcolor[rgb]{0.25,0.63,0.44}{{#1}}}
    \newcommand{\FloatTok}[1]{\textcolor[rgb]{0.25,0.63,0.44}{{#1}}}
    \newcommand{\CharTok}[1]{\textcolor[rgb]{0.25,0.44,0.63}{{#1}}}
    \newcommand{\StringTok}[1]{\textcolor[rgb]{0.25,0.44,0.63}{{#1}}}
    \newcommand{\CommentTok}[1]{\textcolor[rgb]{0.38,0.63,0.69}{\textit{{#1}}}}
    \newcommand{\OtherTok}[1]{\textcolor[rgb]{0.00,0.44,0.13}{{#1}}}
    \newcommand{\AlertTok}[1]{\textcolor[rgb]{1.00,0.00,0.00}{\textbf{{#1}}}}
    \newcommand{\FunctionTok}[1]{\textcolor[rgb]{0.02,0.16,0.49}{{#1}}}
    \newcommand{\RegionMarkerTok}[1]{{#1}}
    \newcommand{\ErrorTok}[1]{\textcolor[rgb]{1.00,0.00,0.00}{\textbf{{#1}}}}
    \newcommand{\NormalTok}[1]{{#1}}
    
    % Additional commands for more recent versions of Pandoc
    \newcommand{\ConstantTok}[1]{\textcolor[rgb]{0.53,0.00,0.00}{{#1}}}
    \newcommand{\SpecialCharTok}[1]{\textcolor[rgb]{0.25,0.44,0.63}{{#1}}}
    \newcommand{\VerbatimStringTok}[1]{\textcolor[rgb]{0.25,0.44,0.63}{{#1}}}
    \newcommand{\SpecialStringTok}[1]{\textcolor[rgb]{0.73,0.40,0.53}{{#1}}}
    \newcommand{\ImportTok}[1]{{#1}}
    \newcommand{\DocumentationTok}[1]{\textcolor[rgb]{0.73,0.13,0.13}{\textit{{#1}}}}
    \newcommand{\AnnotationTok}[1]{\textcolor[rgb]{0.38,0.63,0.69}{\textbf{\textit{{#1}}}}}
    \newcommand{\CommentVarTok}[1]{\textcolor[rgb]{0.38,0.63,0.69}{\textbf{\textit{{#1}}}}}
    \newcommand{\VariableTok}[1]{\textcolor[rgb]{0.10,0.09,0.49}{{#1}}}
    \newcommand{\ControlFlowTok}[1]{\textcolor[rgb]{0.00,0.44,0.13}{\textbf{{#1}}}}
    \newcommand{\OperatorTok}[1]{\textcolor[rgb]{0.40,0.40,0.40}{{#1}}}
    \newcommand{\BuiltInTok}[1]{{#1}}
    \newcommand{\ExtensionTok}[1]{{#1}}
    \newcommand{\PreprocessorTok}[1]{\textcolor[rgb]{0.74,0.48,0.00}{{#1}}}
    \newcommand{\AttributeTok}[1]{\textcolor[rgb]{0.49,0.56,0.16}{{#1}}}
    \newcommand{\InformationTok}[1]{\textcolor[rgb]{0.38,0.63,0.69}{\textbf{\textit{{#1}}}}}
    \newcommand{\WarningTok}[1]{\textcolor[rgb]{0.38,0.63,0.69}{\textbf{\textit{{#1}}}}}
    
    
    % Define a nice break command that doesn't care if a line doesn't already
    % exist.
    \def\br{\hspace*{\fill} \\* }
    % Math Jax compatability definitions
    \def\gt{>}
    \def\lt{<}
    % Document parameters
    \title{1\_introduction\_to\_python}
    
    
    

    % Pygments definitions
    
\makeatletter
\def\PY@reset{\let\PY@it=\relax \let\PY@bf=\relax%
    \let\PY@ul=\relax \let\PY@tc=\relax%
    \let\PY@bc=\relax \let\PY@ff=\relax}
\def\PY@tok#1{\csname PY@tok@#1\endcsname}
\def\PY@toks#1+{\ifx\relax#1\empty\else%
    \PY@tok{#1}\expandafter\PY@toks\fi}
\def\PY@do#1{\PY@bc{\PY@tc{\PY@ul{%
    \PY@it{\PY@bf{\PY@ff{#1}}}}}}}
\def\PY#1#2{\PY@reset\PY@toks#1+\relax+\PY@do{#2}}

\expandafter\def\csname PY@tok@gd\endcsname{\def\PY@tc##1{\textcolor[rgb]{0.63,0.00,0.00}{##1}}}
\expandafter\def\csname PY@tok@gu\endcsname{\let\PY@bf=\textbf\def\PY@tc##1{\textcolor[rgb]{0.50,0.00,0.50}{##1}}}
\expandafter\def\csname PY@tok@gt\endcsname{\def\PY@tc##1{\textcolor[rgb]{0.00,0.27,0.87}{##1}}}
\expandafter\def\csname PY@tok@gs\endcsname{\let\PY@bf=\textbf}
\expandafter\def\csname PY@tok@gr\endcsname{\def\PY@tc##1{\textcolor[rgb]{1.00,0.00,0.00}{##1}}}
\expandafter\def\csname PY@tok@cm\endcsname{\let\PY@it=\textit\def\PY@tc##1{\textcolor[rgb]{0.25,0.50,0.50}{##1}}}
\expandafter\def\csname PY@tok@vg\endcsname{\def\PY@tc##1{\textcolor[rgb]{0.10,0.09,0.49}{##1}}}
\expandafter\def\csname PY@tok@vi\endcsname{\def\PY@tc##1{\textcolor[rgb]{0.10,0.09,0.49}{##1}}}
\expandafter\def\csname PY@tok@vm\endcsname{\def\PY@tc##1{\textcolor[rgb]{0.10,0.09,0.49}{##1}}}
\expandafter\def\csname PY@tok@mh\endcsname{\def\PY@tc##1{\textcolor[rgb]{0.40,0.40,0.40}{##1}}}
\expandafter\def\csname PY@tok@cs\endcsname{\let\PY@it=\textit\def\PY@tc##1{\textcolor[rgb]{0.25,0.50,0.50}{##1}}}
\expandafter\def\csname PY@tok@ge\endcsname{\let\PY@it=\textit}
\expandafter\def\csname PY@tok@vc\endcsname{\def\PY@tc##1{\textcolor[rgb]{0.10,0.09,0.49}{##1}}}
\expandafter\def\csname PY@tok@il\endcsname{\def\PY@tc##1{\textcolor[rgb]{0.40,0.40,0.40}{##1}}}
\expandafter\def\csname PY@tok@go\endcsname{\def\PY@tc##1{\textcolor[rgb]{0.53,0.53,0.53}{##1}}}
\expandafter\def\csname PY@tok@cp\endcsname{\def\PY@tc##1{\textcolor[rgb]{0.74,0.48,0.00}{##1}}}
\expandafter\def\csname PY@tok@gi\endcsname{\def\PY@tc##1{\textcolor[rgb]{0.00,0.63,0.00}{##1}}}
\expandafter\def\csname PY@tok@gh\endcsname{\let\PY@bf=\textbf\def\PY@tc##1{\textcolor[rgb]{0.00,0.00,0.50}{##1}}}
\expandafter\def\csname PY@tok@ni\endcsname{\let\PY@bf=\textbf\def\PY@tc##1{\textcolor[rgb]{0.60,0.60,0.60}{##1}}}
\expandafter\def\csname PY@tok@nl\endcsname{\def\PY@tc##1{\textcolor[rgb]{0.63,0.63,0.00}{##1}}}
\expandafter\def\csname PY@tok@nn\endcsname{\let\PY@bf=\textbf\def\PY@tc##1{\textcolor[rgb]{0.00,0.00,1.00}{##1}}}
\expandafter\def\csname PY@tok@no\endcsname{\def\PY@tc##1{\textcolor[rgb]{0.53,0.00,0.00}{##1}}}
\expandafter\def\csname PY@tok@na\endcsname{\def\PY@tc##1{\textcolor[rgb]{0.49,0.56,0.16}{##1}}}
\expandafter\def\csname PY@tok@nb\endcsname{\def\PY@tc##1{\textcolor[rgb]{0.00,0.50,0.00}{##1}}}
\expandafter\def\csname PY@tok@nc\endcsname{\let\PY@bf=\textbf\def\PY@tc##1{\textcolor[rgb]{0.00,0.00,1.00}{##1}}}
\expandafter\def\csname PY@tok@nd\endcsname{\def\PY@tc##1{\textcolor[rgb]{0.67,0.13,1.00}{##1}}}
\expandafter\def\csname PY@tok@ne\endcsname{\let\PY@bf=\textbf\def\PY@tc##1{\textcolor[rgb]{0.82,0.25,0.23}{##1}}}
\expandafter\def\csname PY@tok@nf\endcsname{\def\PY@tc##1{\textcolor[rgb]{0.00,0.00,1.00}{##1}}}
\expandafter\def\csname PY@tok@si\endcsname{\let\PY@bf=\textbf\def\PY@tc##1{\textcolor[rgb]{0.73,0.40,0.53}{##1}}}
\expandafter\def\csname PY@tok@s2\endcsname{\def\PY@tc##1{\textcolor[rgb]{0.73,0.13,0.13}{##1}}}
\expandafter\def\csname PY@tok@nt\endcsname{\let\PY@bf=\textbf\def\PY@tc##1{\textcolor[rgb]{0.00,0.50,0.00}{##1}}}
\expandafter\def\csname PY@tok@nv\endcsname{\def\PY@tc##1{\textcolor[rgb]{0.10,0.09,0.49}{##1}}}
\expandafter\def\csname PY@tok@s1\endcsname{\def\PY@tc##1{\textcolor[rgb]{0.73,0.13,0.13}{##1}}}
\expandafter\def\csname PY@tok@dl\endcsname{\def\PY@tc##1{\textcolor[rgb]{0.73,0.13,0.13}{##1}}}
\expandafter\def\csname PY@tok@ch\endcsname{\let\PY@it=\textit\def\PY@tc##1{\textcolor[rgb]{0.25,0.50,0.50}{##1}}}
\expandafter\def\csname PY@tok@m\endcsname{\def\PY@tc##1{\textcolor[rgb]{0.40,0.40,0.40}{##1}}}
\expandafter\def\csname PY@tok@gp\endcsname{\let\PY@bf=\textbf\def\PY@tc##1{\textcolor[rgb]{0.00,0.00,0.50}{##1}}}
\expandafter\def\csname PY@tok@sh\endcsname{\def\PY@tc##1{\textcolor[rgb]{0.73,0.13,0.13}{##1}}}
\expandafter\def\csname PY@tok@ow\endcsname{\let\PY@bf=\textbf\def\PY@tc##1{\textcolor[rgb]{0.67,0.13,1.00}{##1}}}
\expandafter\def\csname PY@tok@sx\endcsname{\def\PY@tc##1{\textcolor[rgb]{0.00,0.50,0.00}{##1}}}
\expandafter\def\csname PY@tok@bp\endcsname{\def\PY@tc##1{\textcolor[rgb]{0.00,0.50,0.00}{##1}}}
\expandafter\def\csname PY@tok@c1\endcsname{\let\PY@it=\textit\def\PY@tc##1{\textcolor[rgb]{0.25,0.50,0.50}{##1}}}
\expandafter\def\csname PY@tok@fm\endcsname{\def\PY@tc##1{\textcolor[rgb]{0.00,0.00,1.00}{##1}}}
\expandafter\def\csname PY@tok@o\endcsname{\def\PY@tc##1{\textcolor[rgb]{0.40,0.40,0.40}{##1}}}
\expandafter\def\csname PY@tok@kc\endcsname{\let\PY@bf=\textbf\def\PY@tc##1{\textcolor[rgb]{0.00,0.50,0.00}{##1}}}
\expandafter\def\csname PY@tok@c\endcsname{\let\PY@it=\textit\def\PY@tc##1{\textcolor[rgb]{0.25,0.50,0.50}{##1}}}
\expandafter\def\csname PY@tok@mf\endcsname{\def\PY@tc##1{\textcolor[rgb]{0.40,0.40,0.40}{##1}}}
\expandafter\def\csname PY@tok@err\endcsname{\def\PY@bc##1{\setlength{\fboxsep}{0pt}\fcolorbox[rgb]{1.00,0.00,0.00}{1,1,1}{\strut ##1}}}
\expandafter\def\csname PY@tok@mb\endcsname{\def\PY@tc##1{\textcolor[rgb]{0.40,0.40,0.40}{##1}}}
\expandafter\def\csname PY@tok@ss\endcsname{\def\PY@tc##1{\textcolor[rgb]{0.10,0.09,0.49}{##1}}}
\expandafter\def\csname PY@tok@sr\endcsname{\def\PY@tc##1{\textcolor[rgb]{0.73,0.40,0.53}{##1}}}
\expandafter\def\csname PY@tok@mo\endcsname{\def\PY@tc##1{\textcolor[rgb]{0.40,0.40,0.40}{##1}}}
\expandafter\def\csname PY@tok@kd\endcsname{\let\PY@bf=\textbf\def\PY@tc##1{\textcolor[rgb]{0.00,0.50,0.00}{##1}}}
\expandafter\def\csname PY@tok@mi\endcsname{\def\PY@tc##1{\textcolor[rgb]{0.40,0.40,0.40}{##1}}}
\expandafter\def\csname PY@tok@kn\endcsname{\let\PY@bf=\textbf\def\PY@tc##1{\textcolor[rgb]{0.00,0.50,0.00}{##1}}}
\expandafter\def\csname PY@tok@cpf\endcsname{\let\PY@it=\textit\def\PY@tc##1{\textcolor[rgb]{0.25,0.50,0.50}{##1}}}
\expandafter\def\csname PY@tok@kr\endcsname{\let\PY@bf=\textbf\def\PY@tc##1{\textcolor[rgb]{0.00,0.50,0.00}{##1}}}
\expandafter\def\csname PY@tok@s\endcsname{\def\PY@tc##1{\textcolor[rgb]{0.73,0.13,0.13}{##1}}}
\expandafter\def\csname PY@tok@kp\endcsname{\def\PY@tc##1{\textcolor[rgb]{0.00,0.50,0.00}{##1}}}
\expandafter\def\csname PY@tok@w\endcsname{\def\PY@tc##1{\textcolor[rgb]{0.73,0.73,0.73}{##1}}}
\expandafter\def\csname PY@tok@kt\endcsname{\def\PY@tc##1{\textcolor[rgb]{0.69,0.00,0.25}{##1}}}
\expandafter\def\csname PY@tok@sc\endcsname{\def\PY@tc##1{\textcolor[rgb]{0.73,0.13,0.13}{##1}}}
\expandafter\def\csname PY@tok@sb\endcsname{\def\PY@tc##1{\textcolor[rgb]{0.73,0.13,0.13}{##1}}}
\expandafter\def\csname PY@tok@sa\endcsname{\def\PY@tc##1{\textcolor[rgb]{0.73,0.13,0.13}{##1}}}
\expandafter\def\csname PY@tok@k\endcsname{\let\PY@bf=\textbf\def\PY@tc##1{\textcolor[rgb]{0.00,0.50,0.00}{##1}}}
\expandafter\def\csname PY@tok@se\endcsname{\let\PY@bf=\textbf\def\PY@tc##1{\textcolor[rgb]{0.73,0.40,0.13}{##1}}}
\expandafter\def\csname PY@tok@sd\endcsname{\let\PY@it=\textit\def\PY@tc##1{\textcolor[rgb]{0.73,0.13,0.13}{##1}}}

\def\PYZbs{\char`\\}
\def\PYZus{\char`\_}
\def\PYZob{\char`\{}
\def\PYZcb{\char`\}}
\def\PYZca{\char`\^}
\def\PYZam{\char`\&}
\def\PYZlt{\char`\<}
\def\PYZgt{\char`\>}
\def\PYZsh{\char`\#}
\def\PYZpc{\char`\%}
\def\PYZdl{\char`\$}
\def\PYZhy{\char`\-}
\def\PYZsq{\char`\'}
\def\PYZdq{\char`\"}
\def\PYZti{\char`\~}
% for compatibility with earlier versions
\def\PYZat{@}
\def\PYZlb{[}
\def\PYZrb{]}
\makeatother


    % Exact colors from NB
    \definecolor{incolor}{rgb}{0.0, 0.0, 0.5}
    \definecolor{outcolor}{rgb}{0.545, 0.0, 0.0}



    
    % Prevent overflowing lines due to hard-to-break entities
    \sloppy 
    % Setup hyperref package
    \hypersetup{
      breaklinks=true,  % so long urls are correctly broken across lines
      colorlinks=true,
      urlcolor=urlcolor,
      linkcolor=linkcolor,
      citecolor=citecolor,
      }
    % Slightly bigger margins than the latex defaults
    
    \geometry{verbose,tmargin=1in,bmargin=1in,lmargin=1in,rmargin=1in}
    
    

    \begin{document}
    
    
    \maketitle
    
    

    
    \section{A crash course in Python}\label{a-crash-course-in-python}

\textbf{Authors}: Thierry D.G.A Mondeel, Stefania Astrologo, Ewelina
Weglarz-Tomczak \& Hans V. Westerhoff University of Amsterdam 2016 -
2018

\textbf{Acknowledgements:} This tutorial is inspired in part by a
collection of parts from various notebooks on the web. Some links to the
originals provided below:

\begin{itemize}
\tightlist
\item
  \href{http://ipython-books.github.io/minibook/}{Learning IPython for
  Interactive Computing and Data Visualization, second edition}.**
\item
  https://github.com/rajathkumarmp/Python-Lectures
\item
  http://nbviewer.jupyter.org/urls/bitbucket.org/amjoconn/watpy-learning-to-code-with-python/raw/3441274a54c7ff6ff3e37285aafcbbd8cb4774f0/notebook/Learn\%20to\%20Code\%20with\%20Python.ipynb
\end{itemize}

    {\textbf{Assignment (1 sec):}} Execute the cell below. Don't worry about
what it means.

    \begin{Verbatim}[commandchars=\\\{\}]
{\color{incolor}In [{\color{incolor} }]:} \PY{k+kn}{from} \PY{n+nn}{IPython}\PY{n+nn}{.}\PY{n+nn}{core}\PY{n+nn}{.}\PY{n+nn}{interactiveshell} \PY{k}{import} \PY{n}{InteractiveShell}
        \PY{n}{InteractiveShell}\PY{o}{.}\PY{n}{ast\PYZus{}node\PYZus{}interactivity} \PY{o}{=} \PY{l+s+s2}{\PYZdq{}}\PY{l+s+s2}{all}\PY{l+s+s2}{\PYZdq{}}
\end{Verbatim}


    \subsection{The goal of this notebook}\label{the-goal-of-this-notebook}

The goal here is \textbf{not} to give you a complete introduction to
python. There are separete university courses for this that fill a
semester.

We just want you to be familiar enough to be able to interact with
prewritten Python code to do FBA calculations later on in the tutorial.

    \begin{figure}
\centering
\includegraphics{https://imgs.xkcd.com/comics/python.png}
\caption{title}
\end{figure}

    \subsection{\texorpdfstring{Some properties of
\href{http://www.python.org}{Python}}{Some properties of Python}}\label{some-properties-of-python}

\begin{itemize}
\tightlist
\item
  Python is a modern programming language developed in the early 1990s
  by Guido van Rossum -\textgreater{} A Dutch guy!
  (https://en.wikipedia.org/wiki/Guido\_van\_Rossum)
\item
  Beginner Friendly
\item
  Easy to understand and read
\item
  It is free
\item
  Its use is pervasive in computational biology
\end{itemize}

    \subsection{Printing text in python}\label{printing-text-in-python}

    \begin{Verbatim}[commandchars=\\\{\}]
{\color{incolor}In [{\color{incolor} }]:} \PY{n+nb}{print}\PY{p}{(}\PY{l+s+s2}{\PYZdq{}}\PY{l+s+s2}{The classic view of the central dogma of biology states that }\PY{l+s+se}{\PYZbs{}}
        \PY{l+s+s2}{\PYZsq{}}\PY{l+s+s2}{the coded genetic information hard\PYZhy{}wired into DNA is transcribed into }\PY{l+s+se}{\PYZbs{}}
        \PY{l+s+s2}{individual transportable cassettes, composed of messenger RNA (mRNA); }\PY{l+s+se}{\PYZbs{}}
        \PY{l+s+s2}{each mRNA cassette contains the program for synthesis of a particular }\PY{l+s+se}{\PYZbs{}}
        \PY{l+s+s2}{protein (or small number of proteins).}\PY{l+s+s2}{\PYZsq{}}\PY{l+s+s2}{\PYZdq{}}\PY{p}{)}
\end{Verbatim}


    {\textbf{Assignment (1 min):}} print the name of your favorite protein
below.

    Congratulations! You are now a Python programmer.

    \subsection{Numbers matter in biology}\label{numbers-matter-in-biology}

{\textbf{Assignment (1 min):}} Take a look at the website of
\href{http://bionumbers.hms.harvard.edu/KeyNumbers.aspx}{Bionumbers}

Luckily we can use Python as a calculator.

Below we show some examples. Do you understand what each one does?

\begin{quote}
TIP (Division): In Python 3, \texttt{3\ /\ 2} returns \texttt{1.5}
(floating-point division). \texttt{3\ //\ 2} for integer division.
\end{quote}

    \begin{Verbatim}[commandchars=\\\{\}]
{\color{incolor}In [{\color{incolor} }]:} \PY{l+m+mi}{2} \PY{o}{*} \PY{l+m+mi}{2}
        \PY{l+m+mi}{3}\PY{o}{/}\PY{l+m+mi}{2}
        \PY{l+m+mi}{3}\PY{o}{/}\PY{o}{/}\PY{l+m+mi}{2} \PY{c+c1}{\PYZsh{} integer division}
        \PY{l+m+mi}{6}\PY{o}{\PYZpc{}}\PY{k}{2}
        \PY{l+m+mi}{8}\PY{o}{\PYZpc{}}\PY{k}{3}
        \PY{l+m+mi}{2}\PY{o}{*}\PY{o}{*}\PY{l+m+mi}{5}
\end{Verbatim}


    Python's built-in mathematical operators include \texttt{+}, \texttt{-},
\texttt{*}, \texttt{**}, for exponentiation, \texttt{/} for division,
\texttt{//} for integer division (without rest), \texttt{\%} for
division but giving you the rest.

    {\textbf{Assignment (3 min):}} According to
\href{http://bionumbers.hms.harvard.edu/KeyNumbers.aspx}{Bionumbers} e.
coli has a volume of up to \(5~\mu m^3\) and yeast has a volume of up to
\(160~\mu m^3\)

Calculate in the cell below how many times bigger the volume of yeast is
compared to e. coli.

    {\textbf{Assignment (3 min):}} Using
\href{http://bionumbers.hms.harvard.edu/KeyNumbers.aspx}{Bionumbers}
find the "average protein diameter" and the "diameter of a yeast cell".
Then calculate how many "average proteins" you could theoretically lay
side by side in a yeast cell. \textbf{Neglect any aspects of folding or
interactions.} Just divide one number by the other.

    \subsection{Variables}\label{variables}

Variables form a fundamental concept of any programming language. A
variable has a name and a value. Here is how to create a new variable in
Python:

    \begin{Verbatim}[commandchars=\\\{\}]
{\color{incolor}In [{\color{incolor} }]:} \PY{n}{avogadro} \PY{o}{=} \PY{l+m+mf}{6e23}
        \PY{n+nb}{print}\PY{p}{(}\PY{l+s+s1}{\PYZsq{}}\PY{l+s+s1}{How many molecules are contained in a mole? Answer:}\PY{l+s+s1}{\PYZsq{}}\PY{p}{,}\PY{n}{avogadro}\PY{p}{,}\PY{l+s+s1}{\PYZsq{}}\PY{l+s+s1}{molecules.}\PY{l+s+s1}{\PYZsq{}}\PY{p}{)}
\end{Verbatim}


    And here is how to use an existing variable:

    \begin{Verbatim}[commandchars=\\\{\}]
{\color{incolor}In [{\color{incolor} }]:} \PY{n+nb}{print}\PY{p}{(}\PY{l+s+s1}{\PYZsq{}}\PY{l+s+s1}{How many molecules in 2 moles? Answer:}\PY{l+s+s1}{\PYZsq{}}\PY{p}{,}\PY{l+m+mi}{2}\PY{o}{*}\PY{n}{avogadro}\PY{p}{,}\PY{l+s+s1}{\PYZsq{}}\PY{l+s+s1}{molecules}\PY{l+s+s1}{\PYZsq{}}\PY{p}{)}
\end{Verbatim}


    {\textbf{Assignment (2 min):}} * Look up the molar mass of water
(\(H_2O\)) * Use the avogadro number to calculate and print the number
of molecules in 1L of water

    \subsection{Getting help in the
notebook}\label{getting-help-in-the-notebook}

When you want to know what a command or function does you can type a
question mark in front of the command.

    \begin{Verbatim}[commandchars=\\\{\}]
{\color{incolor}In [{\color{incolor} }]:} \PY{k+kn}{import} \PY{n+nn}{math} \PY{c+c1}{\PYZsh{} load some common mathematical operations}
        \PY{n}{math}\PY{o}{.}\PY{n}{sqrt}
\end{Verbatim}


    When a command requires input arguments, as for math.sqrt(put a number
here), a handy keyboard shortcut is Shift-Tab. A tooltip will light up
showing you the various arguments you can pass to the command.

{\textbf{Assignment (2 min):}} Ask for help on the "math.sqrt" command
using the Shift-Tab method described above. Start by typing math.sqrt
below and then press Shift-Tab.

    \subsection{Finding functions that belong to an object or are part of a
module: What kind of methods does math
contain?}\label{finding-functions-that-belong-to-an-object-or-are-part-of-a-module-what-kind-of-methods-does-math-contain}

When we import a library like "math", which we did at the top of the
notebook, this library will contain many different functions like sqrt
above. If you want to find out which ones type "math." (notice the dot!)
and then press "TAB".

{\textbf{Assignment (3 min):}} Use the TAB key to find out some other
math functions and play around with them maybe look at the help file.

    There are different types of variables. Here, we have used a number
(more precisely, an \textbf{integer}). Other important types include
\textbf{floating-point numbers} to represent real numbers,
\textbf{strings} to represent text, and \textbf{booleans} to represent
\texttt{True/False} values. Here are a few examples:

    \begin{Verbatim}[commandchars=\\\{\}]
{\color{incolor}In [{\color{incolor} }]:} \PY{n}{somefloat} \PY{o}{=} \PY{l+m+mf}{3.1415}
        \PY{n}{sometext} \PY{o}{=} \PY{l+s+s1}{\PYZsq{}}\PY{l+s+s1}{pi is about}\PY{l+s+s1}{\PYZsq{}}  \PY{c+c1}{\PYZsh{} You can also use double quotes.}
        \PY{n+nb}{print}\PY{p}{(}\PY{n}{sometext}\PY{p}{,} \PY{n}{somefloat}\PY{p}{)}  \PY{c+c1}{\PYZsh{} Display several variables.}
        \PY{n}{I\PYZus{}am\PYZus{}true} \PY{o}{=} \PY{k+kc}{False}
        \PY{n}{I\PYZus{}am\PYZus{}true} 
\end{Verbatim}


    Note how we used the \texttt{\#} character to write \textbf{comments}.
Whereas Python discards the comments completely, adding comments in the
code is important when the code is to be read by other humans (including
yourself in the future).

    {\textbf{Assignment (3 min):}} Make your own piece of text, i.e. your
name and age and print it to the screen. Do not just write a string a
text, make your age a variable like pi in the example above.

    \subsection{Murphy's law: (In a tutorial) Anything that can go wrong
will go
wrong}\label{murphys-law-in-a-tutorial-anything-that-can-go-wrong-will-go-wrong}

When you write something Python doesn't understand it throws an
exception and tries to explain what went wrong, but it can only speak in
a broken Pythonesque english.

Let's see some examples by running these code blocks. This is helpful
later on because you will likely encounter (and produce) some errors.

    \begin{Verbatim}[commandchars=\\\{\}]
{\color{incolor}In [{\color{incolor} }]:} \PY{n}{gibberish}
\end{Verbatim}


    \begin{Verbatim}[commandchars=\\\{\}]
{\color{incolor}In [{\color{incolor} }]:} \PY{o}{*}\PY{n}{adsflf\PYZus{}}
\end{Verbatim}


    \begin{Verbatim}[commandchars=\\\{\}]
{\color{incolor}In [{\color{incolor} }]:} \PY{n+nb}{print}\PY{p}{(}\PY{l+s+s1}{\PYZsq{}}\PY{l+s+s1}{Hello}\PY{l+s+s1}{\PYZsq{}}
\end{Verbatim}


    \begin{Verbatim}[commandchars=\\\{\}]
{\color{incolor}In [{\color{incolor} }]:} \PY{l+m+mi}{1}\PY{n}{\PYZus{}my\PYZus{}variable\PYZus{}starting\PYZus{}with\PYZus{}a\PYZus{}number} \PY{o}{=} \PY{l+m+mi}{1}
\end{Verbatim}


    \begin{Verbatim}[commandchars=\\\{\}]
{\color{incolor}In [{\color{incolor} }]:} \PY{l+m+mi}{2000} \PY{o}{/} \PY{l+m+mi}{0}
\end{Verbatim}


    Python tries to tell you where it stopped understanding, but in the
above examples, each program is only 1 line long.

It also tries to show you where on the line the problem happened with
caret ("\^{}").

Finally it tells you the type of thing that went wrong, (NameError,
SyntaxError, ZeroDivisionError) and a bit more information like "name
'gibberish' is not defined" or "unexpected EOF while parsing".

Unfortunately you might not find "unexpected EOF while parsing" too
helpful. EOF stands for End of File, but what file? What is parsing?
Python does it's best, but it does take a bit of time to develop a knack
for what these messages mean. If you run into an error you don't
understand please ask a tutor.

    \subsection{Types of variables in
python}\label{types-of-variables-in-python}

When you define a variable in python it has a type. Above we dealt with
numbers which are of type ... Below we briefly introduce the other types
you might see in the rest of the tutorial.

Simply put there is text, i.e. strings, and two kinds of containers,
e.g. lists and dictionaries.

    \subsubsection{The Written Word, i.e.
strings}\label{the-written-word-i.e.-strings}

Numbers are great... but most of our day to day computing needs involves
text, from emails to tweets to documents. Or in biology: DNA sequences,
chemical formulas, hyperlinks between databases etc.

We have already seen a couple strings in Python. Programmers call text
\emph{strings} because they are weird like that. From now on we will
only refer to strings, but we just mean pieces of text inside our code.

    \begin{Verbatim}[commandchars=\\\{\}]
{\color{incolor}In [{\color{incolor} }]:} \PY{l+s+s2}{\PYZdq{}}\PY{l+s+s2}{Hello, World!}\PY{l+s+s2}{\PYZdq{}}
\end{Verbatim}


    Strings are surrounded by quotes. Without the quotes Hello by itself
would be viewed as a variable name.

You can use either double quotes (") or single quotes (') for
text/strings. As we saw before we can also save text in variables.

Let's use strings with variables!

    \begin{Verbatim}[commandchars=\\\{\}]
{\color{incolor}In [{\color{incolor} }]:} \PY{n}{your\PYZus{}name} \PY{o}{=} \PY{l+s+s2}{\PYZdq{}}\PY{l+s+s2}{James Watson}\PY{l+s+s2}{\PYZdq{}}
        \PY{n+nb}{print}\PY{p}{(}\PY{l+s+s2}{\PYZdq{}}\PY{l+s+s2}{Hello,}\PY{l+s+s2}{\PYZdq{}}\PY{p}{,}\PY{n}{your\PYZus{}name}\PY{p}{)}
\end{Verbatim}


    Strings in Python are a bit more complicated because the operations on
them aren't just + and * (though those are valid operations).

    \subsection{Dot notation and object oriented
programming}\label{dot-notation-and-object-oriented-programming}

Python, like many programming languages, supports Object Oriented
Programming or OOP for short. In this paradigm, we approach ideas as
Objects much as we do in the real world. Each Object is an instance of a
Class or a type of object. Such an object may have certain properties or
function that can be applied to them.

So what does all this have to do with dot notation? Dot notation allows
us to tell a instance of a class to use one of the functions inside that
class. That is why we access the sqrt function from the math module with
the dot. And why we access the 'upper' function of a string as
my\_string.upper()

    {\textbf{Assignment (2 min):}} Below, after your\_string, type a dot and
press Tab. You will get a list of function you can apply to the string.
Pick any command that seems interesting to you.

\textbf{Note} that you have to define a string first to be able to get
help. So first execute the cell so that the variable your\_string is
known. Then using the Tab key find one of your choice.

\textbf{Note:} The point here is just to get you comfortable with the
dot notation and finding functions and properties of objects.

    \begin{Verbatim}[commandchars=\\\{\}]
{\color{incolor}In [{\color{incolor} }]:} \PY{n}{your\PYZus{}string} \PY{o}{=} \PY{l+s+s1}{\PYZsq{}}\PY{l+s+s1}{something}\PY{l+s+s1}{\PYZsq{}}
        \PY{n}{your\PYZus{}string}
\end{Verbatim}


    \subsection{Lists}\label{lists}

    A list contains a sequence of items. You can concisely instruct Python
to perform repeated actions on the elements of a list. Let's first
create a list of numbers:

    \begin{Verbatim}[commandchars=\\\{\}]
{\color{incolor}In [{\color{incolor} }]:} \PY{n}{items} \PY{o}{=} \PY{p}{[}\PY{l+m+mi}{1}\PY{p}{,} \PY{l+m+mi}{3}\PY{p}{,} \PY{l+m+mi}{0}\PY{p}{,} \PY{l+m+mi}{4}\PY{p}{,} \PY{l+m+mi}{1}\PY{p}{]}
\end{Verbatim}


    Note the syntax we used to create the list: square brackets
\texttt{{[}{]}}, and commas \texttt{,} to separate the items.

The \emph{built-in} function \texttt{len()} returns the number of
elements in a list:

    \begin{Verbatim}[commandchars=\\\{\}]
{\color{incolor}In [{\color{incolor} }]:} \PY{n+nb}{len}\PY{p}{(}\PY{n}{items}\PY{p}{)}
\end{Verbatim}


    We can also access individual elements in the list, using the following
syntax:

    \begin{Verbatim}[commandchars=\\\{\}]
{\color{incolor}In [{\color{incolor} }]:} \PY{n}{items}\PY{p}{[}\PY{l+m+mi}{0}\PY{p}{]}
        
        \PY{n}{items}\PY{p}{[}\PY{o}{\PYZhy{}}\PY{l+m+mi}{1}\PY{p}{]} 
\end{Verbatim}


    Note that indexing starts at \texttt{0} in Python: the first element of
the list is indexed by \texttt{0}, the second by \texttt{1}, and so on.
Also, \texttt{-1} refers to the last element, \texttt{-2}, to the
penultimate element, and so on.

    The same syntax can be used to alter elements in the list:

    \begin{Verbatim}[commandchars=\\\{\}]
{\color{incolor}In [{\color{incolor} }]:} \PY{n}{items}\PY{p}{[}\PY{l+m+mi}{1}\PY{p}{]} \PY{o}{=} \PY{l+m+mi}{9}
        \PY{n}{items}
\end{Verbatim}


    We can access sublists with the following syntax:

    \begin{Verbatim}[commandchars=\\\{\}]
{\color{incolor}In [{\color{incolor} }]:} \PY{n}{items}\PY{p}{[}\PY{l+m+mi}{1}\PY{p}{:}\PY{l+m+mi}{3}\PY{p}{]}
\end{Verbatim}


    Here, \texttt{1:3} represents a \textbf{slice} going from element
\texttt{1} \emph{included} (this is the second element of the list) to
element \texttt{3} \emph{excluded}. Thus, we get a sublist with the
second and third element of the original list. The
first-included/last-excluded asymmetry leads to an intuitive treatment
of overlaps between consecutive slices. Also, note that a sublist refers
to a dynamic \emph{view} of the original list, not a copy; changing
elements in the sublist automatically changes them in the original list.

    {\textbf{Assignment (3 min):}} In the code cell below make a list of the
first 10 prime numbers: https://en.wikipedia.org/wiki/Prime\_number. Use
the sum() function to figure out the sum of the first 10 prime numbers.

    {\textbf{Assignment (1 min):}} Print the second-to-last prime number
from your list to the screen.

    \subsection{Dictionaries}\label{dictionaries}

Dictionaries contain key-value pairs. They are extremely useful and
common. They allow you to map, or point, \textbf{keys} to
\textbf{values}. In the example below the letters a,b,c are now pointing
to the numbers 1,2,3.

For a flux balance analysis application of a dictionary, you can think
of a dictionary that points each of the reactions in a network to its
flux in the FBA solution.

You can access the \textbf{value} a certain \textbf{key} points to with
square bracket notation:

    \begin{Verbatim}[commandchars=\\\{\}]
{\color{incolor}In [{\color{incolor} }]:} \PY{n}{my\PYZus{}dict} \PY{o}{=} \PY{p}{\PYZob{}}\PY{l+s+s1}{\PYZsq{}}\PY{l+s+s1}{a}\PY{l+s+s1}{\PYZsq{}}\PY{p}{:} \PY{l+m+mi}{1}\PY{p}{,} \PY{l+s+s1}{\PYZsq{}}\PY{l+s+s1}{b}\PY{l+s+s1}{\PYZsq{}}\PY{p}{:} \PY{l+m+mi}{2}\PY{p}{,} \PY{l+s+s1}{\PYZsq{}}\PY{l+s+s1}{c}\PY{l+s+s1}{\PYZsq{}}\PY{p}{:} \PY{l+m+mi}{3}\PY{p}{\PYZcb{}}
        \PY{n+nb}{print}\PY{p}{(}\PY{l+s+s1}{\PYZsq{}}\PY{l+s+s1}{a:}\PY{l+s+s1}{\PYZsq{}}\PY{p}{,} \PY{n}{my\PYZus{}dict}\PY{p}{[}\PY{l+s+s1}{\PYZsq{}}\PY{l+s+s1}{a}\PY{l+s+s1}{\PYZsq{}}\PY{p}{]}\PY{p}{)}
\end{Verbatim}


    \begin{Verbatim}[commandchars=\\\{\}]
{\color{incolor}In [{\color{incolor} }]:} \PY{n+nb}{list}\PY{p}{(}\PY{n}{my\PYZus{}dict}\PY{o}{.}\PY{n}{keys}\PY{p}{(}\PY{p}{)}\PY{p}{)}
\end{Verbatim}


    The keys in a dictionary can be anything including numbers

    \begin{Verbatim}[commandchars=\\\{\}]
{\color{incolor}In [{\color{incolor} }]:} \PY{n}{my\PYZus{}dict} \PY{o}{=} \PY{p}{\PYZob{}}\PY{l+m+mi}{18}\PY{p}{:} \PY{l+m+mi}{1}\PY{p}{,} \PY{l+m+mi}{23}\PY{p}{:} \PY{l+m+mi}{2}\PY{p}{,} \PY{l+m+mi}{0}\PY{p}{:} \PY{l+m+mi}{3}\PY{p}{\PYZcb{}}
        \PY{n}{my\PYZus{}dict}\PY{p}{[}\PY{l+m+mi}{18}\PY{p}{]}
\end{Verbatim}


    {\textbf{Assignment (1 min):}} Make your own dictionary. For yourself
and a friend add a 'key' to the dictionary, the name of the person, and
a value, the age of the person. Then print the dictionary to the screen.

    \subsection{for loops}\label{for-loops}

    We can run through all elements of a list using a \texttt{for} loop:

    \begin{Verbatim}[commandchars=\\\{\}]
{\color{incolor}In [{\color{incolor} }]:} \PY{n}{a\PYZus{}list} \PY{o}{=} \PY{p}{[}\PY{l+m+mi}{1}\PY{p}{,}\PY{l+m+mi}{2}\PY{p}{,}\PY{l+m+mi}{3}\PY{p}{,}\PY{l+m+mi}{4}\PY{p}{,}\PY{l+m+mi}{5}\PY{p}{,}\PY{l+m+mi}{6}\PY{p}{]}
        \PY{k}{for} \PY{n}{number} \PY{o+ow}{in} \PY{n}{a\PYZus{}list}\PY{p}{:}
            \PY{n}{number}
\end{Verbatim}


    \begin{itemize}
\tightlist
\item
  Note that the for loop steps in sequence through the numbers in the
  list.
\item
  Every loop, the variable number is assigned the value of the next
  number in the list
\item
  You may call this variable number whatever you wish. As long as you
  also change it in the third line
\end{itemize}

As a more complex example, we can also loop over the \textbf{keys} of a
dictionary.

    \begin{Verbatim}[commandchars=\\\{\}]
{\color{incolor}In [{\color{incolor} }]:} \PY{n}{genome\PYZus{}sizes} \PY{o}{=} \PY{p}{\PYZob{}}\PY{l+s+s1}{\PYZsq{}}\PY{l+s+s1}{e. coli}\PY{l+s+s1}{\PYZsq{}}\PY{p}{:}\PY{l+m+mi}{5}\PY{p}{,}\PY{l+s+s1}{\PYZsq{}}\PY{l+s+s1}{yeast}\PY{l+s+s1}{\PYZsq{}}\PY{p}{:}\PY{l+m+mi}{12}\PY{p}{,}\PY{l+s+s1}{\PYZsq{}}\PY{l+s+s1}{human}\PY{l+s+s1}{\PYZsq{}}\PY{p}{:}\PY{l+m+mf}{2.9e3}\PY{p}{\PYZcb{}} \PY{c+c1}{\PYZsh{} dictionary}
        
        \PY{n+nb}{print}\PY{p}{(}\PY{l+s+s1}{\PYZsq{}}\PY{l+s+s1}{A list of genome\PYZhy{}sizes in \PYZsh{}Mbp:}\PY{l+s+s1}{\PYZsq{}}\PY{p}{)}
        
        \PY{k}{for} \PY{n}{organism} \PY{o+ow}{in} \PY{n}{genome\PYZus{}sizes}\PY{o}{.}\PY{n}{keys}\PY{p}{(}\PY{p}{)}\PY{p}{:}
            \PY{n+nb}{print}\PY{p}{(}\PY{n}{organism}\PY{p}{,}\PY{n}{genome\PYZus{}sizes}\PY{p}{[}\PY{n}{organism}\PY{p}{]}\PY{p}{)}
\end{Verbatim}


    There are several things to note here:

    \begin{itemize}
\tightlist
\item
  The \texttt{for\ organism\ in\ genome\_sizes.keys()} syntax means that
  a temporary variable named \texttt{organism} is created at every
  iteration. This variable contains the value of every item in the list,
  one at a time.
\item
  Note the colon \texttt{:} at the end of the \texttt{for} statement.
  Forgetting it will lead to a syntax error!
\item
  The \texttt{print} statement will be executed for all items in the
  list.
\item
  Note the four spaces before \texttt{print}: this is called the
  \textbf{indentation}. You will find more details about indentation in
  the next subsection.
\end{itemize}

    {\textbf{Assignment (3 min):}} Write your own for loop that prints each
element of your list of 10 prime numbers divided by 2 separately to the
screen.

    \subsection{List comprehensions: for loops in one
line}\label{list-comprehensions-for-loops-in-one-line}

    Python supports a concise syntax to perform a given operation on all
elements of a list using for loops:

    \begin{Verbatim}[commandchars=\\\{\}]
{\color{incolor}In [{\color{incolor} }]:} \PY{n}{items} \PY{o}{=} \PY{p}{[}\PY{l+m+mi}{1}\PY{p}{,}\PY{l+m+mi}{2}\PY{p}{,}\PY{l+m+mi}{3}\PY{p}{,}\PY{l+m+mi}{4}\PY{p}{,}\PY{l+m+mi}{5}\PY{p}{,}\PY{l+m+mi}{6}\PY{p}{]}
        \PY{n}{squares} \PY{o}{=} \PY{p}{[}\PY{n}{item} \PY{o}{*} \PY{n}{item} \PY{k}{for} \PY{n}{item} \PY{o+ow}{in} \PY{n}{items}\PY{p}{]}
        \PY{n}{squares}
\end{Verbatim}


    This is called a \textbf{list comprehension}. A new list is created
here; it contains the squares of all numbers in the list. This concise
syntax leads to highly readable and \emph{Pythonic} code.

    {\textbf{Assignment (3 min):}} Write a list comprehension that
calculates the square of each of the first 10 prime numbers. Start from
the example above but loop over the list of prime numbers you defined
above.

    \section{Now that you are a Python genius we can move on to flux balance
analysis!}\label{now-that-you-are-a-python-genius-we-can-move-on-to-flux-balance-analysis}


    % Add a bibliography block to the postdoc
    
    
    
    \end{document}
